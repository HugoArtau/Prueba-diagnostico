\documentclass[11pt, addpoints, answers]{exam}
\usepackage[spanish]{babel}
\usepackage{amsmath}
\usepackage{amssymb}
\usepackage{graphicx}
\usepackage{tabularx}
\usepackage{ragged2e}
\usepackage{geometry}
\usepackage{booktabs}
\usepackage{comment} % paquete que te permite hacer comentarios de varias lineas de codigo a la vez.

% --- CONFIGURACIÓN DE PUNTUACIÓN EN ESPAÑOL  ---
% Definimos cómo se muestran los puntos: "punto" para singular y "puntos" para plural.

\pointpoints{punto}{puntos}

% --- AJUSTE DE MARGENES ---
\geometry{
	hmargin={1in, 1in},
	vmargin={1.2in, 1in},
}

% --- CONFIGURACIÓN DE PÁGINA Y ENCABEZADO FINAL ---
\pagestyle{headandfoot}

% 1. Definición Ultra-Robusta del Encabezado (soluciona Overfull \hbox)
\renewcommand{\firstpageheadrule}{%
	\dimen0=\tabcolsep
	\makebox[\textwidth]{%
		\hspace{-\dimen0}\makebox[\textwidth+2\dimen0]{%
			\textbf{Nombre del estudiante:} \enspace\hrulefill \hspace{2em}
			\textbf{Grupo:} \enspace\hrulefill
		}%
	}\vspace{1ex}\hrule
}

% 2. Define el contenido del encabezado de la universidad (centrado)
\firstpageheader{}{\centering\textbf{Examen Diagnóstico de Matemática}\\\textbf{Universidad Estatal Guayaquil}}{}

% 3. Define el pie de página
\firstpagefooter{}{Página \thepage\ de \numpages}{}
\runningheader{Diagnóstico Matemática}{}{Pág. \thepage}
\runningfooter{}{}{}

\begin{document}
% --- NUEVA SECCIÓN PARA DATOS DEL ALUMNO ---
% Usamos \makebox para alinear el Nombre a la izquierda y el Grupo a la derecha.
\makebox[\textwidth]{
	\textbf{Nombre:} \enspace\hrulefill \hspace{4em}  
	\textbf{Grupo:} \enspace\hrulefill 
}


	
	\vspace{1cm}
	
	% -------------------------------------------------------------
	% COMIENZA EL EXAMEN (Todas las preguntas)
	% -------------------------------------------------------------
	\begin{questions}
		
		% Pregunta 1 : Ecuaciones
		
		\question[1] \textbf{Resuelve las siguientes ecuaciones.} 
		
		\begin{parts}
			\part $x^2 - 4 = 0$
			\part $3x+\sqrt{6x+10}=35$ %x=9
			\part $2^{3x-5}=1024$
			\part $\dfrac{5x-7}{2}-3x=\dfrac{2x+7}{3}-14$ % x=-7 comprobar
		\end{parts}
		\begin{solution}
		\begin{parts}
			\part $x_{1}=$\hspace{2in} $x_{2}=$
			\part $x=$
			\part $x=$
			\part $x=$
	\end{parts}	
		\end{solution}
		
		% PREGUNTA 2: ÁNGULOS (Múltiple Opción)
		
		\question[1] \textbf{¿Qué ángulos son iguales al ángulo 6 de la figura?}
		\begin{center}
			\includegraphics[width=0.4\textwidth]{image_243168.png}
		\end{center}
		\begin{choices}
			\choice 2, 3 y 5
			\choice 2, 3 y 7
			\choice 5, 7 y 2
			\choice 1, 4 y 7
		\end{choices}
		
		% PREGUNTA 3: SÍMBOLOS DE AGRUPACIÓN
		
		\question[1] \textbf{Elimine los símbolos de agrupación y reduzca términos semejantes.}  
		\begin{parts}
			\part $17 - 7(3x - 4)$
			\part $2(5x - 4y) - (7x + y)$
			\part $x - [7 - 3(2x - 4)]$
			\part $(2x - 3y) - 4(x - 5y)$
		\end{parts}
		
		% PREGUNTA 4: FACTORIZACIÓN
		
		\question[1] \textbf{Factorizar los siguientes polinomios.}  
		\begin{parts}
			\part $4(x-3)^2 + 6(x-3)$
			\part $9a^2 - 4$
			\part $x^2 + 8x + 15$
			\part $2x^2 + 9x + 4$
		\end{parts}
		
		% PREGUNTA 5: FRACCIONES ALGEBRAICAS (VERTICAL Y CON PUNTUACIÓN AJUSTADA)
		
		\question[1] \textbf{Si sumamos las fracciones $\frac{4}{x+2} + \frac{2x}{x+2}$ dará por resultado:}
		
		  	
		\begin{oneparchoices} % CAMBIADO a choices para opciones verticales
			\choice $\dfrac{6x}{x+2}$
			\choice $\dfrac{4+2x}{x+2}$
			\choice $\dfrac{4+2x}{(x+2)^2}$
			\choice 2
		\end{oneparchoices}
		
		\vspace{1cm}
		\begin{comment}
	
	
		\question[1] \textbf{Si sumamos $\frac{x}{x+3} + \frac{2}{x-2}$ su resultado será}:
	
		\begin{choices} % CAMBIADO a choices para opciones verticales
			\choice $\frac{2x}{(x+3)(x-2)}$
			\choice $\frac{2x}{2x+5}$
			\choice $\frac{x^2+6}{(x+3)(x-2)}$
			\choice $\frac{1}{6}$
		\end{choices}
			\end{comment}
		
		% PREGUNTA 6: SISTEMA DE ECUACIONES
		
		\question[1] \textbf{El conjunto solución del sistema $\begin{cases} 3x+2y=7 \\ 2x+5y=12 \end{cases}$ es:}
		
		\begin{oneparchoices}
			\textbf{\choice} $(1, 2)$
			\textbf{\choice} $(-1, 2)$
			\textbf{\choice} $(1, -2)$
			\textbf{\choice} $(2, 1)$
		\end{oneparchoices}
		
		% PREGUNTA 7: RECTAS
		
		\question[1] \textbf{Considerando la siguiente gráfica, la ecuación de la recta sería:}
		\begin{center}
			\includegraphics[width=0.6\textwidth]{image_248acc.png}
		
		\end{center}
		\begin{oneparchoices}
			\textbf{\choice} $y-3 = \dfrac{4}{3}x+6$
			\textbf{\choice} $y = 5x-2$
			\textbf{\choice} $y-2 = -\dfrac{4}{3}x - \dfrac{4}{3}$
			\textbf{\choice} $y = -\dfrac{1}{3}x+2$
		\end{oneparchoices}
		
			\vspace{1.2cm}
		
		% PREGUNTA 8: PROPIEDADES DE POTENCIAS
		
		\question[1] \textbf{¿Cuál de las siguientes igualdades es correcta?}
		\begin{checkboxes}
			\choice $7^2 \cdot 7^4 \cdot 7 = 7^6$
			\choice $7^2 \cdot 7^{-1} \cdot 7^3 = 7^4$
			\choice $7^2 + 7^4 + 7 = 21^7$
			\choice $7^2 + 7^{-1} + 7^3 = 7^3$
		\end{checkboxes}
		
		% PREGUNTA 9: FUNCIÓNes lineales
		\question[1] \textbf{Selecciona cuál de las siguientes reglas de correspondencia pertenece a la función representada.}
		\begin{center}
			\includegraphics[width=0.5\textwidth]{image_249683.png}
		\end{center}
		\begin{checkboxes}
			\choice $f(x) = \sqrt{x-5}-1$
			
			\choice $f(x) = \sqrt{x-1}-3$
			\choice $f(x) = \sqrt{x+1}-3$
			\choice $f(x) = \sqrt{x-1}+3$
		\end{checkboxes}
\question[1]\textbf{En el $\triangle$ MNP calcula la altura relativa al lado $\overline{MN}$ si sus vértices son: M(4,-2), N(-8,7) y P(2,7)}.	% recuerda que: Hallar la altura pedida no es mas que calcular la distancia del punto P a la recta que contiene al lado MND desarrollo del ejercicio ver Matemática 11no grado Ejemplo 6 pagina 81.
		
		% PREGUNTA 10: RELACIONAR COLUMNAS (Tabla)
		\question[1] \textbf{Escribe en la Columna B los resultados correspondientes a cada una de las operaciones de la Columna A.}
		
		
		\begin{tabularx}{\textwidth}{|>{\RaggedRight\arraybackslash}X|X|}
			\hline
			\textbf{Columna A (Operación)} & \textbf{Columna B (Resultado)} \\
			\hline
			$\left(\left(\left(\frac{1}{2}\right)^3\right)^0\right)^{-2}$ & \\
			\hline
			$(6x^2+13x-5) \div (3x-1)$ & \\
			\hline
			$\sqrt[3]{9x^6 \cdot x^8}$ & \\
			\hline
			$\frac{3}{4} = \frac{9}{?}$ & \\
			\hline
			$\log(100) - \log(10)$ & \\
			\hline
		\end{tabularx}
		
		\vspace{0,5cm}
		
		% Pregunta 12: Geometría Plana
		
	\question[1] \textbf{En la figura tenemos que AC bisectriz del $\angle$ DAB. Además $\angle$ DEB=110$^\circ$, $\angle$ ADE=70$^\circ$, CBA=60$^\circ$. % ACB=100
		  Halla la amplitud de $\angle$ ACB}
\begin{center}
	\includegraphics[width=0.7\linewidth]{fig1}
\end{center}
\begin{comment}
\question[1] Si $l_{1}$ es paralela a $l_{2}$. Calcule el valor de $x$. %Respuesta: 29 grados
\begin{figure}
	\centering
	\includegraphics[width=0.7\linewidth]{fig2}
\end{figure}	
\end{comment}


\question[1] \textbf{Si $l_{1}$ es paralela a $l_{2}$. Calcule el valor de $x$.}
\begin{figure}
	\centering
	\includegraphics[width=0.7\linewidth]{fig3}
\end{figure}


%Pregunta 14: Ecuaciones logarítmicas
		  
\question[1]\textbf{Dada la ecuación $\log_2{(x^{2}+1)}-\log_4{x^{2}}=1$. La suma de sus soluciones es:}

\begin{oneparcheckboxes}
	
	\choice 0 %Respuesta correcta
	\choice 1
	\choice 2
	\choice 3
\end{oneparcheckboxes}
\question[1]\textbf{Halle el valor de $x$ en la siguiente ecuación $9\cdot27^{x}=27$}

%Pregunta 13: ecuaciones  exponencial

\begin{oneparcheckboxes}
	
	\choice $\dfrac{1}{2}$ 
	\choice $\dfrac{1}{3}$ 
	\choice 3
	\choice $\dfrac{1}{9}$ 
\end{oneparcheckboxes}	

% Preguntas 16: Dominio de funciones.
\question[1]\textbf{Determina el dominio de las siguientes funciones:	 } 

\begin{parts}
\part $ f(x)=\sqrt{\dfrac{x-2}{x+3}}$
\part $ g(x)=\dfrac{2x}{x^{2}-3x+2}$
\part $ h(x)=\dfrac{1}{1+x^2}$
\part $ k(x)=\log_{x}{2}$ %Respuesta: Dom(f)={x pertenece a R/x>0 y xdiferente de 1} codigo latex:$$\text{Dom}(f) = \{x \in \mathbb{R} \mid x > 0 \land x \neq 1\}$$ 0 en intervalo  $$\text{Dom}(f) = (0, 1) \cup (1, \infty)$$
 \part $ p(x)=\log{(x+3)}$	
\end{parts}

% Pregunta 17: Inecuaciones.
\question[1]\textbf{ Resuelve las siguientes inecuaciones:}

\begin{parts}
\part $2 \leq 5-3x\leq 11$
\part $5x-4(x+5)<x-24$
\part $4x+1\ge3-5x>10-7x$
\part $x^{2}-x-2\ge0$
\part $x^{2}-x-6>0$
\part $x(6x-1)<(3x+5)(2x-7)$
\end{parts}

\question[1]\textbf{Seleccione la solución correcta a la siguiente inecuación:}$7x^{2}+21x-28<0$
\begin{parts}
\part $S=(-4,1)$
\part $S=(-\infty,-4)\cup (1,+\infty)$
\part $S=(-\infty,1)\cup (4,+\infty)$
\part $S=(-1,4)$
\end{parts}

\end{questions}
\end{document}